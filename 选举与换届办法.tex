\chapter{北京大学青年天文学会(学生社团)选举与换届办法}

本办法于2019年5月26日线上骨干成员会议讨论通过,自2019年5月26日起正式施行,原《北京大学青年天文学会换届选举办法》自行废止。

为规范北京大学青年天文学会(学生社团)选举与换届流程,根据《北京大学青年天文学会(学生社团)章程》制定本办法。

\section{主席团成员的选举}

\begin{enumerate}
    \item 北京大学青年天文学会(学生社团)主席团成员由选举大会选举产生。选举大会一般在每学年夏季学期倒数第二次骨干成员会议上进行。选举方式为民主投票选举。
    
    \item 满足以下条件中任意一项的骨干成员具有选举权,即成为选举人:
    
    \begin{enumerate}
        \item 在本学期参加过三分之一(含)以上次数的骨干成员会议;
        \item 由当届主席团认定在本学期参与过较多的骨干成员事务;
        \item 由当届主席团认定在本学年对学会做出过重大贡献。
    \end{enumerate}
    
    \item 满足以下全部条件的骨干成员具有被选举权:
    
    \begin{enumerate}
        \item 出席选举大会;
        \item 在本学期参加过二分之一(含)以上次数的骨干成员会议;
        \item 先前未担任过主席团成员职位;
        \item 不违反《北京大学学生社团管理办法》中对社团负责人的相关要求。
    \end{enumerate}
    
    \item 主席团成员候选人按以下程序产生:
    
    \begin{enumerate}
        \item 具有被选举权的骨干成员进行自荐;
        \item 若自荐人数超过三人(含),则所有自荐的骨干成员成为候选人;反之,则所有具有被选举权的骨干成员全部成为候选人。
    \end{enumerate}
    
    \item 选举大会的召开应满足以下条件:
    
    \begin{enumerate}
        \item 超过四分之三(含)的选举人到会;
        \item 经过指导教师、挂靠单位的批准。
    \end{enumerate}
    
    \item 选举大会按以下程序进行:
    
    \begin{enumerate}
        \item 现任主席团成员确认选举人到会人数满足本办法第五条中的规定,会长宣布开会;
        \item 现任主席团成员作总结发言;
        \item 根据本办法第四条产生候选人,会长告知候选人各主席团成员的主要工作;
        \item 各候选人作竞选发言,每位候选人的发言时间不得超过90秒;
        \item 发放选票;
        \item 选举人对候选人进行无记名投票,每位选举人在选票上写三个候选人的姓名,多写、少写或写候选人以外的人的选票均视作废票;
        \item 回收选票;
        \item 唱票;
        \item 根据投票结果产生新一届主席团成员:
        
        \begin{enumerate}
            \item 若得票超过到会选举人的半数(含)的候选人人数超过三人(含),则得票数前三名的候选人即成为新一届主席团成员;若第三名与第四名得票数相同,则所有得票数与第三名相同的候选人重新进行一轮投票并按照投票结果进行排位,直至第三名与第四名得票数不再相同,前三名候选人即成为新一届主席团成员;
            \item 若得票超过到会选举人的半数(含)的候选人人数不及三人,则得票数超过半数(含)的候选人成为新一届主席团成员,并从得票数未超过半数的候选人中按得票数从高到低取主席团空缺人数的两倍人数,重新进行一轮投票,直至有得票数超过半数(含)的候选人补足空缺人数为止。
            \item 每进行一轮投票前,所有参与投票的候选人均需进行一轮发言,每位候选人的发言时间不得超过30秒。
        \end{enumerate}
        
        \item 新一届主席团成员进行职位分配:
        
        \begin{enumerate}
            \item 新一届主席团成员按照得票数名次与决出轮次进行排序,原则上第一名成为新一届会长,第二名成为新一届团支书,第三名成为新一届副会长;
            \item 若新一届主席团成员对职位分配存在异议,则可以进行不超过三分钟的内部协商,若达成共识,则除新一届主席团成员外的选举人需对协商结果进行表决,若以三分之二多数通过,则按照协商结果分配职位;若未达成共识或协商结果表决未通过,则按照第1项规定进行职位分配。
        \end{enumerate}

        \item 新一届会长、团支书、副会长分别发言并分别与现任会长、团支书、副会长合影;
    \end{enumerate}

\end{enumerate}

\section{主席团的换届}

\begin{enumerate}[resume]
    \item 新一届主席团成员的上任时间为秋季学期的第一次骨干成员会议,秋季学期社团注册工作由新一届主席团负责。
    
    \item 自新一届主席团成员选出之日起至上任之日止,现任主席团成员应协助新一届主席团成员完成过渡交接等工作。现任主席团成员有义务对新一届主席团成员提供必要的培训、引导与支持。
    
    \item 新一届主席团成员上任后,主席团成员应及时与指导教师、挂靠单位与管理单位联系,告知学会换届情况,并接受指导教师、挂靠单位与管理单位的指导、管理与监督。
\end{enumerate}

\section{理事长的选举与换届}

\begin{enumerate}[resume]
    \item 理事长的选举与换届应在选举大会召开前完成。
    
    \item 理事长的选举与换届应满足以下条件:超过四分之三(含)的选举人到场;经过指导教师、挂靠单位的批准。
    
    \item 理事长的选举与换届按以下程序进行:
    
    \begin{enumerate}
        \item 现任理事长确认选举人到会人数满足本办法第十一条中的规定;
        \item 现任理事长在所有担任过主席团成员职位的骨干成员中考察并提名一人成为新一届理事长候选人;
        \item 发放选票;
        \item 选举人对新一届理事长候选人进行无记名表决;
        \item 回收选票;
        \item 唱票;
        \item 若三分之二(含)以上的到场选举人支持理事长候选人,则表决通过;反之则表决不通过,现任理事长需重新按本程序提名另一人进行表决,直至表决通过。
        \item 表决通过后新一届理事长立即上任。
    \end{enumerate}
    
\end{enumerate}

\section{部长团的任命}

\begin{enumerate}[resume]
    \item 新一届主席团应在秋季学期的第一次骨干成员会议上任命部长团。
    
    \item 部长团由各部门部长与副部长组成,部门必须设部长,但可以不设副部长。
    
    \item 部长团成员应满足以下条件:
    
    \begin{enumerate}
        \item 部长团成员为骨干成员;
        \item 主席团成员不得兼任部长,但可以兼任不多于一个部门的副部长;
        \item 主席团成员外的任一骨干成员可以兼任部长与副部长,但不得兼任多于一个部门的部长,不得兼任多于两个部门(不含)的副部长, 不得兼任同一部门的部长和副部长。
    \end{enumerate}
    
    \item 部长团成员在满足本办法第十五条中的规定的条件下无任期限制。
\end{enumerate}

\section{附则}

\begin{enumerate}[resume]
    \item 若在选举与换届过程中出现任何本办法内未规定的情形,由所有到场选举人本着民主、公正的原则协商进行处理。
    
    \item 本办法的解释权归北京大学青年天文学会(学生社团)所有,指导教师、挂靠单位和北京大学学生课外活动指导中心有权对本章程的解释提出建议。
\end{enumerate}
