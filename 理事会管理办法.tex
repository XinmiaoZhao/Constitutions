\chapter{北京大学青年天文学会(学生社团)理事会管理办法}

本办法于2019年x月xx日骨干成员会议讨论通过,自2019年x月xx日起正式施行,原《北京大学青年天文学会理事会制度》自行废止。

为规范北京大学青年天文学会(学生社团)。。。,根据《北京大学青年天文学会(学生社团)章程》制定本办法。

\section{总则}

\begin{enumerate}
    \item 北京大学青年天文学会(学生社团)理事会是学会的内部知道机构(顾问咨询?)。理事会的职责是对学会的日常运行情况提供意见与建议,在学会出现突发重大事件时负责协商并维持学会的基本运转。
    
    \item 北京大学青年天文学会(学生社团)理事会设置理事长,职责是组织理事会的建设,协调理事会成员间的意见。
    
    \item ?常务顾问至少一人,必须为历届骨干。常务顾问要求按时参加每周例会,连续两次不出席者主席团可以酌情免去其常务顾问头衔。
\end{enumerate}

\section{理事会成员}

\begin{enumerate}[resume]
    \item 满足下列条件的北京大学青年天文学会(学生社团)会员可以自愿接受邀请成为理事会成员:\label{item:council}
    \begin{enumerate}
        \item 离任主席团成员、部长团成员;
        \item 成为骨干成员满两年的离任骨干成员;
        \item 本科三年级及以上且成为骨干成员满一年的离任骨干成员。
    \end{enumerate}

    \item 理事会成员身份在会员具有北京大学青年天文学会(学生社团)会籍期间内有效。
    
    \item 理事会成员除享有《北京大学青年天文学会(学生社团)章程》中规定的会员权利外,还享有下列权利:
    \begin{enumerate}
        \item 了解学会内部运行情况的权利;
        \item 指导学会工作的权利;
        \item !在发生突发重大事件时。。。的权利。
    \end{enumerate}

    \item 理事会成员除应当履行《北京大学青年天文学会(学生社团)章程》中规定的会员义务外,还应履行下列义务:
    \begin{enumerate}
        \item 了解学会动态的义务;
        \item 为主席团提供有价值的建议与必要的支持的义务;
    \end{enumerate}

    \item 理事会成员分为常务理事和普通理事。常务理事为骨干成员,享有参加骨干成员会议等的骨干成员权利,也应履行相应骨干成员义务;普通理事非骨干成员。
    
    \item 若常务理事被免除骨干成员身份,则自动变为普通理事。
    
    \item 普通理事可以自愿申请再次成为骨干成员,则自动变为常务理事,其骨干成员身份视为重新申请之学期取得。
    
    \item 满足下列任一条件则免除理事会成员身份:
    \begin{enumerate}
        \item 不再具有北京大学青年天文学会(学生社团)会员身份;
        \item 主动申请免除理事会成员身份,经理事长同意;
        \item 严重违反学会章程或各项办法条例,严重损害学会合法权益,经理事长同意并经骨干成员会议表决通过;
        \item 违反法律法规规章或严重违反校规校纪,经理事长同意并经骨干成员会议表决通过;
        \item 其他应予免除理事会成员身份的情况。
    \end{enumerate}

\end{enumerate}

\section{理事长}

\begin{enumerate}[resume]
    \item 理事长的选举与换届办法见《北京大学青年天文学会(学生社团)选举与换届办法》。
    
    \item 理事长在上任后负责邀请满足第\ref{item:council}条规定的会员加入理事会,统计接受邀请的会员名单,并传达至主席团。
    
    \item 
\end{enumerate}

\section{附则}

\begin{enumerate}[resume]
    \item 本办法的解释权归北京大学青年天文学会(学生社团)所有,指导教师、挂靠单位和北京大学学生课外活动指导中心有权对本办法的解释提出建议。
\end{enumerate}
