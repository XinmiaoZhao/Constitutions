\chapter{北京大学青年天文学会(学生社团)突发重大事件处理办法}

本办法于2019年x月xx日骨干成员会议讨论通过,自2019年x月xx日起正式施行。

为规范北京大学青年天文学会(学生社团)。。。,根据《北京大学青年天文学会(学生社团)章程》制定本办法。

\section{总则}

\begin{enumerate}
    \item !“突发重大事件”指的是社团管理层的各种变故产生的导致学会无法正常运行的需要立即处理的突发紧急事件。
    
    \item 本办法中规定的“社团管理层”包括学会主席团成员、部长团成员与理事长(理事会成员?)。
    
    \item 本办法中规定的与社团管理层成员“取得联系”包括下列途径:
    \begin{enumerate}
        \item 通过微信等即时聊天工具联系;
        \item 通过电话或短信联系;
        \item 走访寝室联系;
        \item 前往教室或办公室寻找;
        \item 通过会长近亲友间接联系;
        \item 其他有可能与社团管理层成员取得联系的方式。
    \end{enumerate}
    
    \item 出现下列任一情形,即为本办法中规定的“造成较为严重后果”:
    \begin{enumerate}
        \item 因触犯《北京大学学生社团管理办法》,被管理单位处以批评警告、通报批评或暂停活动限期整改的处分;
        \item 在活动过程中遗失或损坏器材,造成学会3000元以上20000元以下的经济损失;
        \item 在外出活动过程中遗失人员后又找回且人员未受伤,但被未参加活动的人员得知;%确认受伤情况?
        \item 在外出活动过程中遗失人员后又找回且人员受轻微伤或轻伤;
        \item 在活动过程中对器材操作或管理不慎,造成其他人员轻微伤或轻伤(因伤者自己操作或管理不慎的除外);
        \item 在活动过程中监管不慎,间接造成其他人员因自己操作不慎重伤或死亡;
        \item ?因管理不善,对社团名誉造成严重损害;
        \item 经社团管理层成员的三分之二以上的多数同意的其他造成较为严重后果的情形。
    \end{enumerate}
    
    \item 出现下列任一情形,即为本办法中规定的“造成极为严重后果”:
    \begin{enumerate}
        \item 因触犯《北京大学学生社团管理办法》,被管理单位处以限期换届或勒令解散的处分;
        \item 在活动过程中遗失或损坏器材,造成学会20000元及以上的经济损失;
        \item 在外出活动过程中遗失人员后又找回且人员受重伤或死亡;
        \item 在外出活动过程中遗失人员后宣告人员失踪;
        \item 在活动过程中对器材操作或管理不慎,造成其他成员重伤或死亡(因伤者或逝者自己操作或管理不慎的除外);
        \item 经社团管理层成员的三分之二以上的多数同意的其他造成极为严重后果的情形。
    \end{enumerate}
\end{enumerate}

\section{会长}

\begin{enumerate}[resume]
    \item 如果自一位骨干成员无法联系到会长起,48小时内超过五位骨干成员(含)尝试联系会长但无人取得联系,即可宣布会长失联。宣布会长失联后,社团管理层应努力联系会长直到取得联系。
    
    \item 满足下列情况可以宣布会长失踪:
    
    \begin{enumerate}
        \item 宣布会长失联后连续120小时内,以高于每12小时一次的频率尝试联系会长均无法取得联系;
        \item 会长因事故或其他原因下落不明,经判定短时间无法联系;
        \item 会长因触犯法律、法规或其他管理条例,被公安机关拘留或逮捕;
        \item 经社团管理层成员的三分之二以上的多数同意的其他能够证明168小时内无法联系到会长的情况。
    \end{enumerate}

    \item 满足下列情况可以宣布会长不能履行职务:
    
    \begin{enumerate}
        \item 宣布会长失踪后连续168小时内,以高于每24小时一次的频率尝试联系会长均无法取得联系;
        \item 会长因触犯法律、法规或其他管理条例,受到刑事或行政处罚,经研究决定需要剥夺会长职务的;
        \item 会长一意孤行,做出严重有损社团名誉事情的;
        \item 会长一意孤行,做出令至少80\%的社团管理层成员反对的事情且不加悔改的;
        \item 会长一意孤行,做出令至少60\%的社团管理层成员反对的事情,造成严重后果且不加悔改的;
        \item 会长一意孤行,想解散社团的;
        \item 会长怠工,学期内连续2周不举行活动,遭到50\%以上社团管理层的反对,经调解无效的;
        \item 会长连续2周无故不参加例会的;
        \item 会长连续3周不参加例会,导致社团活动无法正常进行的;
        \item 会长因疾病、逝世等突发因素无法继续履行职务的;
        \item 经社团管理层成员研究,能够证明会长不能履行职务的其他情况(此条需要除会长外社团管理层超过2/3成员通过)。
    \end{enumerate}
    
    \item 如宣告会长失踪,或宣告会长失联且期间有重大活动的,由理事会成员指定一名主席团的其他成员代行会长之职,超过24个小时未指定的,由团支书代行会长之职。如24个小时未得到团支书回复或团支书明确表示不想代行会长之职,按以下顺序代行会长之职:
    
    \begin{enumerate}
        \item 副会长;
        \item 理事长;
        \item 除理事长外,理事会成员中曾经担任过会长的人(年级从低到高);
        \item 除理事长外,理事会成员中曾经担任过团支书的人(年级从低到高);
        \item 除理事长外,理事会成员中曾经担任过副会长的人(年级从低到高);
        \item 除理事长外,理事会成员中一般理事(年级从高到低);
        \item 各部部长(顺序赵鑫淼你自己定吧)。
    \end{enumerate}
    
    \item 会长可以宣布辞职,但应有正当理由,正当理由包括:
    
    \begin{enumerate}
        \item 因重大疾病不能履行会长之职;
        \item 因成绩原因不能履行会长之职;
        \item 因交换、休学或退学,不能履行会长之职;
        \item 因办理活动出现重大失误,受到校团委通报批评及以上处分,引咎辞职;
        \item 其他经社团管理层成员研究可以辞职的情况。
    \end{enumerate}

    \item 会长宣布辞职,骨干研究后,2/3骨干决定同意辞职的,会长可以辞职并择期重新选举。
    
    \item 如会长宣布辞职或被宣布不能履行职务的情况,应按照第九条的方法选出临时会长,并在两周之内举行换届大会,重新选举会长或主席团成员。
\end{enumerate}

\section{团支书}


\begin{enumerate}[resume]
    \item 如骨干内超过5名成员在96个小时内均无法因公事联系到团支书,即可宣布团支书失联。
    
    \item 宣布团支书失联后,社团管理层应努力联系团支书直到联系成功。如遇以下情况,可以宣布团支书失踪:
    
    \begin{enumerate}
        \item 宣布团支书失联后连续144个小时,以至少每24个小时一次的频率联系团支书均宣告失败;
        \item 团支书因事故或其他原因下落不明,经判定短时间无法联系;
        \item 团支书因触犯法律、法规或其他管理条例,被公安机关逮捕;
        \item 经社团管理层成员研究,其他能够证明168小时内无法联系到团支书的情况(此条需要除团支书外社团管理层超过2/3成员通过)。
    \end{enumerate}

    \item 如遇以下情况,可以宣布团支书不能履行职务。
    
    \begin{enumerate}
        \item 宣布团支书失踪后连续10天,以至少每24个小时一次的频率联系团支书均宣告失败;
        \item 团支书因触犯法律、法规或其他管理条例,受到刑事或行政处罚,经研究决定需要剥夺团支书职务的;
        \item 团支书一意孤行,做出严重有损社团名誉事情的;
        \item 团支书一意孤行,不听取会长意见且与50\%以上的社团管理层意见相左,最后造成重大损失的;
        \item 团支书连续2月无故不参加例会的;
        \item 团支书连续3月不参加例会,导致社团活动无法正常进行的;
        \item 团支书因疾病、逝世等突发因素无法继续履行职务的;
        \item 经社团管理层成员研究,能够证明团支书不能履行职务的其他情况(此条需要除团支书外社团管理层超过2/3成员通过)。
    \end{enumerate}
    
    \item 如宣告团支书失踪,或宣告团支书失联,且期间有重大活动的,则由理事会成员指定一名主席团的其他成员代行团支书之职,超过24小时未指定的,按以下顺序代行团支书之职:
    
    \begin{enumerate}
        \item 副会长;
        \item 理事长;
        \item 除理事长外,理事会成员中曾经担任过会长的人(年级从低到高);
        \item 除理事长外,理事会成员中曾经担任过团支书的人(年级从低到高);
        \item 除理事长外,理事会成员中曾经担任过副会长的人(年级从低到高);
        \item 除理事长外,理事会成员中一般理事(年级从高到低);
        \item 各部部长(顺序赵鑫淼你自己定吧)。
    \end{enumerate}
    
    \item 团支书可以宣布辞职,但应有正当理由,正当理由包括:
    
    \begin{enumerate}
        \item 因重大疾病不能履行团支书之职;
        \item 因成绩原因不能履行团支书之职;
        \item 因交换、休学或退学,不能履行会长之职;
        \item 因办理活动出现重大失误,受到校团委限期整改及以上处分,引咎辞职;
        \item 其他经社团管理层成员研究可以辞职的情况。
    \end{enumerate}

    \item 团支书宣布辞职,需要满足第十八条中所列情形之一。社团管理层成员研究后,2/3同意辞职的,团支书可以辞职并择期重新选举。
    
    \item 如团支书宣布辞职或被宣布不能履行职务的情况,应按照第十七条的方法选出临时团支书。如有必要,可以在两周之内举行换届大会,重新选举团支书或团支书和副会长。
\end{enumerate}

\section{副会长}


\begin{enumerate}[resume]
    \item 如遇以下情况,可以宣布副会长不能履行职务。
    
    \begin{enumerate}
        \item 在30天内,第一周以至少每12个小时一次,其余时间以每24个小时一次的频率联系副会长均宣告失败;
        \item 副会长因触犯法律、法规或其他管理条例,受到刑事或行政处罚,经研究决定需要剥夺会长职务的;
        \item 副会长一意孤行,做出严重有损社团名誉事情的;
        \item 副会长一意孤行,不听取会长和团支书意见且与50\%以上的社团管理层意见相左,最后造成重大损失的;
        \item 副会长连续2月无故不参加例会的;
        \item 副会长连续3月不参加例会,导致社团活动无法正常进行的;
        \item 副会长因疾病、逝世等突发因素无法继续履行职务的;
        \item 经社团管理层成员研究,能够证明副会长不能履行职务的其他情况(此条需要除副会长外社团管理层超过2/3成员通过)。
    \end{enumerate}
    
    \item 如副会长失去联系且期间有需要副会长的重大活动的,由理事会成员指定一名主席团的其他成员代行副会长之职,按以下顺序代行副会长之职:
    
    \begin{enumerate}
        \item 理事长;
        \item 除理事长外,理事会成员中曾经担任过会长的人(年级从低到高);
        \item 除理事长外,理事会成员中曾经担任过团支书的人(年级从低到高);
        \item 除理事长外,理事会成员中曾经担任过副会长的人(年级从低到高);
        \item 除理事长外,理事会成员中一般理事(年级从高到低);
        \item 各部部长(顺序赵鑫淼你自己定吧)。
    \end{enumerate}
    
    \item 副会长可以宣布辞职,但应有正当理由,正当理由包括:
    
    \begin{enumerate}
        \item 因重大疾病不能履行副会长之职;
        \item 因交换、休学或退学,不能履行副会长之职;
        \item 因办理活动出现重大失误,受到校团委通报批评及以上处分,引咎辞职;
        \item 其他经社团管理层成员研究可以辞职的情况(此条需要除副会长外社团管理层超过2/3成员通过)。
    \end{enumerate}

    \item 副会长宣布辞职,需要满足第二十三条中所列情形之一。骨干研究决定批准辞职的,副会长可以辞职,如有必要,择期重新选举。
    
    \item 如副会长宣布辞职或被宣布不能履行职务的情况,应按照第九条的方法选出临时副会长。如有必要,可以在两周之内举行换届大会,重新选举副会长。
\end{enumerate}

\section{理事会}


\begin{enumerate}[resume]
    \item 如遇主席团发生严重错误的情况,理事会可以介入调解至驳回主席团决定。
    
    \begin{enumerate}
        \item 主席团三人不和,争论不休,或主席团怠工,导致活动无法顺利进行的;
        \item 主席团两人或三人一意孤行,造成较为严重后果且不思悔改,变本加厉的;
        \item 经理事会成员研究,犯下其他较为严重的错误的(此条需要理事会超过2/3成员通过)。
    \end{enumerate}
    
    \item 如遇主席团发生极为严重错误的情况,理事会可以暂时剥夺主席团的权力,时间从一周到一个月,情节严重者可以举行换届。
    
    \begin{enumerate}
        \item 主席团想解散社团的;
        \item 主席团三人不和,争论不休,或怠工并造成极为严重后果且不思悔改的;
        \item 经理事会成员研究,犯下其他极为严重的错误的(此条需要理事会超过2/3成员通过)。
    \end{enumerate}
    
    \item 如遇以下情况,可以宣布理事长不能履行职务。
    
    \begin{enumerate}
        \item 在30天内,第一周以至少每12个小时一次,其余时间以每24个小时一次的频率联系理事长均宣告失败;
        \item 理事长因触犯法律、法规或其他管理条例,受到刑事或行政处罚,经研究决定需要剥夺理事长职务的;
        \item 理事长一意孤行,做出严重有损社团名誉事情的;
        \item 理事长一意孤行,不听取主席团意见且与50\%以上的社团管理层意见相左,最后造成重大损失的;
        \item 理事长一意孤行,在没有听取理事会建议的情况下擅自行使驳回决定、剥夺权力或举行换届权力的;
        \item 理事长连续3月不参加例会,导致社团活动无法正常进行的;
        \item 理事长因疾病、逝世等突发因素无法继续履行职务的;
        \item 经社团管理层成员研究,能够证明理事长不能履行职务的其他情况(此条需要除理事长外社团管理层超过2/3成员通过)。
    \end{enumerate}
    
    \item 理事长可以宣布辞职,但应有正当理由,正当理由包括:
    
    \begin{enumerate}
        \item 因重大疾病不能履行理事长之职;
        \item 因成绩原因不能履行理事长之职;
        \item 因交换、休学或退学等原因不能履行理事长之职;
        \item 因办理活动出现重大失误,受到校团委限期整改及以上处分,引咎辞职;
        \item 其他经社团管理层成员研究可以辞职的情况(此条需要除理事长外社团管理层超过2/3成员通过)。
    \end{enumerate}

    \item 理事长宣布辞职,需要满足第二十九条中所列情形之一并指定继承人。若未指定由理事会中在青天会担任骨干时间最久的人员担当理事长。
\end{enumerate}

\section{附则}

\begin{enumerate}[resume]
    \item 本办法的解释权归北京大学青年天文学会(学生社团)所有,指导教师、挂靠单位和管理单位有权对本办法的解释提出建议。
\end{enumerate}

