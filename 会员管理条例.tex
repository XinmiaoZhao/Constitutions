\chapter{北京大学青年天文学会(学生社团)会员管理条例}

本条例于2019年x月xx日骨干成员会议讨论通过,自2019年x月xx日起正式施行,原《北京大学青年天文学会会员管理条例》自行废止。

为规范北京大学青年天文学会(学生社团)会员管理,根据《北京大学青年天文学会(学生社团)章程》制定本条例。

\section{总则}

\begin{enumerate}
    \item 北京大学青年天文学会(学生社团)实行会员制,加入学会的学生称为北京大学青年天文学会(学生社团)会员,具有北京大学青年天文学会(学生社团)会籍。
    
    \item 学会为全体会员提供良好的业余天文交流平台,并提供必要的帮助与支持。
    
    \item !北京大学青年天文学会(学生社团)会员管理由学会组织部负责。
\end{enumerate}

\section{会员分类}

\begin{enumerate}[resume]
    \item 北京大学青年天文学会(学生社团)会员分为普通会员、高级会员和骨干成员。
    
    \item 所有新加入北京大学青年天文学会(学生社团)的会员均为普通会员,普通会员可以通过考核(主要以高级会员考试的形式)晋升为高级会员,普通会员和高级会员都可申请成为骨干成员。
\end{enumerate}

\section{会员的加入与退出}

\begin{enumerate}[resume]
    \item 北京大学青年天文学会(学生社团)会员必须是具有北京大学正式学籍的在校学生,会员身份在会员身为具有北京大学正式学籍的在校学生期间内有效。
    
    \item 满足下列条件的具有北京大学正式学籍的在校学生可以自愿申请加入北京大学青年天文学会(学生社团):
    \begin{enumerate}
        \item 认同北京大学青年天文学会(学生社团)的宗旨;
        \item 愿意参加北京大学青年天文学会(学生社团)组织的活动;
        \item 遵守学会章程以及各项办法条例。
    \end{enumerate}
    
    \item 会员加入北京大学青年天文学会(学生社团)按下列程序进行:
    \begin{enumerate}
        \item 在每学期社团招新活动现场申请加入或在学期中任何时间申请加入;
        \item 按本办法第\ref{item:membership_fee}条规定缴纳会费;
        \item 学会向新会员颁发普通会员证。
    \end{enumerate}

    \item 满足下列任一条件则免除会员身份:
    \begin{enumerate}
        \item 因毕业、退学等原因失去北京大学正式学籍;
        \item 主动申请免除会员身份;
        \item 严重违反学会章程或各项办法条例,严重损害学会合法权益,经骨干成员会议表决通过;
        \item 违反法律法规规章或严重违反校规校纪,经骨干成员会议表决通过;
        \item 其他应予免除会员身份的情况。
    \end{enumerate}
\end{enumerate}

\section{骨干成员}

\begin{enumerate}[resume]
    \item 会员在加入学会后任何时间都可以自愿申请,经主席团批准后成为骨干成员。骨干成员身份在会员具有北京大学青年天文学会(学生社团)会籍期间内有效。
    
    \item 满足下列任一条件则免除骨干成员身份:
    \begin{enumerate}
        \item 不再具有北京大学青年天文学会(学生社团)会员身份;
        \item 在当前学期内连续三次无故缺席骨干成员会议;
        \item 主动申请免除骨干成员身份,经主席团同意;
        \item 严重违反学会章程或各项办法条例,严重损害学会合法权益,经骨干成员会议表决通过;
        \item 违反法律法规规章或严重违反校规校纪,经骨干成员会议表决通过;
        \item 其他应予免除骨干成员身份的情况。
    \end{enumerate}

    \item 会员若在骨干成员身份被免除后重新申请成为骨干成员,其骨干成员身份视为重新申请之学期取得。
    
    \item 骨干成员除享有《北京大学青年天文学会(学生社团)章程》中规定的会员权利外,还享有下列权利:
    \begin{enumerate}
        \item 参加骨干成员会议与参与骨干成员内部活动的权利;
        \item 选举权与表决权;
        \item 被选举权。
    \end{enumerate}

    \item 骨干成员的选举权与被选举权之规定见《北京大学青年天文学会(学生社团)选举与换届办法》。
    
    \item 在骨干成员会议上,骨干成员表决权的获得按照下列办法进行:\label{item:right_for_vote}
    \begin{enumerate}
        \item 对于当前学期成为骨干成员的骨干成员,若参加过当前学期三次(含)以上且三分之一(含)以上次数的骨干成员会议则具有表决权;
        \item 对于非当前学期成为骨干成员的骨干成员,若本学期骨干成员会议举办次数超过三次(含),则参加过当前学期三分之一(含)以上次数的骨干成员会议者具有表决权;反之则一律具有表决权。
    \end{enumerate}

    \item 骨干成员除应当履行《北京大学青年天文学会(学生社团)章程》中规定的会员义务外,还应履行下列义务:
    \begin{enumerate}
        \item 参加骨干成员会议的义务;
        \item 积极参与学会各项活动筹备、组织的义务。
    \end{enumerate}
    
    \item !北京大学青年天文学会(学生社团)的主席团、部长团成员与理事长、常务理事必须为骨干成员。
\end{enumerate}

\section{高级会员}

\begin{enumerate}[resume]
    \item 高级会员是具有较高天文知识水平的学会会员,高级会员身份在会员具有北京大学青年天文学会(学生社团)会籍期间内有效。
    
    \item 所有普通会员均可通过高级会员考试成为高级会员。每位会员参加考试次数不限。
    
    \item 原则上高级会员考试每学期举办一次,主席团有权决定增加或减少考试次数。
    
    \item 高级会员考试的命题、监考、阅卷工作由高级会员考试命题组负责。
    
    \item 高级会员考试命题组由主席团中的高级会员负责组建。若主席团成员均非高级会员,则理事会应推举一位高级会员组建高级会员考试命题组。命题组全部成员必须为高级会员。
    
    \item 高级会员考试分为试题考试和观测考试两部分,考试内容为基本的业余天文知识,达到或超过命题组划定的分数线的普通会员即可成为高级会员。
    
    \item 高级会员除享有《北京大学青年天文学会(学生社团)章程》中规定的会员权利外,还享有下列权利:
    \begin{enumerate}
        \item 借用学会设备器材的权利。
    \end{enumerate}

    \item 高级会员除应当履行《北京大学青年天文学会(学生社团)章程》中规定的会员义务外,还应履行下列义务:
    \begin{enumerate}
        \item 维护学会设备器材的义务;
        \item 积极参与会内科普活动的义务。
    \end{enumerate}

    \item 满足下列任一条件则免除高级会员身份:
    \begin{enumerate}
        \item 不再具有北京大学青年天文学会(学生社团)会员身份;
        \item 主动申请免除高级会员身份,经主席团同意;
        \item 丢失或严重损坏学会设备器材,造成学会财产严重损失,经骨干成员会议表决通过;
        \item 严重违反学会章程或各项办法条例,严重损害学会合法权益,经骨干成员会议表决通过;
        \item 违反法律法规规章或严重违反校规校纪,经骨干成员会议表决通过;
        \item 其他应予免除高级会员身份的情况。
    \end{enumerate}
\end{enumerate}

\section{会费与会员证}

\begin{enumerate}[resume]
    \item 会员在加入北京大学青年天文学会(学生社团)时,需要缴纳会费。会员缴纳的会费纳入学会日常运营经费。
    
    \item 现行会费标准为每人25元。\label{item:membership_fee}

    \item 会员证是北京大学青年天文学会(学生社团)会员身份的凭证,学会在会员加入学会时向其颁发普通会员证,在会员成为高级会员时向其颁发高级会员证。
    
    \item 普通会员证在会员拥有北京大学青年天文学会(学生社团)会籍期间内有效,高级会员证在会员具有高级会员身份期间内有效。
    
    \item 会员如果丢失会员证,可以向学会申请补办。
\end{enumerate}

\section{附则}

\begin{enumerate}[resume]        
    \item 本条例的解释权归北京大学青年天文学会(学生社团)所有,指导教师、挂靠单位和管理单位有权对本条例的解释提出建议。
\end{enumerate}
