\chapter{北京大学青年天文学会(学生社团)会员管理条例}

本条例于2019年x月xx日骨干成员会议讨论通过,自2019年x月xx日起正式施行,原《北京大学青年天文学会会员管理条例》自行废止。

为规范北京大学青年天文学会(学生社团)会员管理,根据《北京大学青年天文学会(学生社团)章程》制定本条例。

\section{总则}

\begin{enumerate}
    \item 北京大学青年天文学会(学生社团)实行会员制,加入学会的学生称为北京大学青年天文学会(学生社团)会员。
    
    \item 学会为全体会员提供良好的业余天文交流平台,并提供必要的帮助与支持。
    
    \item 北京大学青年天文学会(学生社团)会员管理由学会组织部负责。
\end{enumerate}

\section{会员的加入与退出}

\begin{enumerate}[resume]
    \item 北京大学青年天文学会(学生社团)会员必须是具有北京大学正式学籍的全日制在校学生,会员身份在会员身为具有北京大学正式学籍的全日制在校学生期间内有效。
    
    \item 加入北京大学青年天文学会(学生社团)的条件:
    
    \begin{enumerate}
        \item 自愿加入北京大学青年天文学会(学生社团);
        \item 遵守北京大学青年天文学会(学生社团)的规定;
        \item 身为具有北京大学正式学籍的全日制在校学生。
    \end{enumerate}
    
    \item 加入北京大学青年天文学会(学生社团)的程序:
    
    \begin{enumerate}
        \item 每学期招新时现场申请加入或在学期中任何时间申请加入;
        \item 每位新入会会员缴纳20元会费,成为北京大学青年天文学会(学生社团)会员;
        \item 北京大学青年天文学会(学生社团)向新会员颁发会员证。
    \end{enumerate}

    \item 会员若因毕业、退学等原因失去北京大学全日制在校学生身份,则将被视为自动退出北京大学青年天文学会(学生社团)。

    \item 北京大学青年天文学会(学生社团)会员如有严重违反本章程的行为,学会有开除其会员资格的权力。 

\end{enumerate}

\section{会员分类}

\begin{enumerate}[resume]
    \item 北京大学青年天文学会(学生社团)会员分为:普通会员、高级会员和骨干成员。
    
    \item 所有新加入北京大学青年天文学会(学生社团)的会员均为普通会员,普通会员可以通过考核(主要以高级会员考试的形式)晋升为高级会员,普通会员和高级会员都可申请成为骨干成员。
\end{enumerate}

\section{骨干成员}

\begin{enumerate}[resume]
    \item 会员在加入学会后任何时间都可以自愿申请成为骨干成员。
    
    \item 骨干成员除享有《北京大学青年天文学会(学生社团)章程》中规定的会员权利外,还享有下列权利:
    \begin{enumerate}
        \item 参加骨干成员会议与参与骨干成员内部活动的权利;
        \item 选举权与表决权;
        \item 被选举权。
    \end{enumerate}

    \item 骨干成员除应当履行《北京大学青年天文学会(学生社团)章程》中规定的会员义务外,还应履行下列义务:
    \begin{enumerate}
        \item 参加骨干成员会议的义务;
        \item 积极参与学会各项活动筹备、组织的义务。
    \end{enumerate}
    
    \item 北京大学青年天文学会(学生社团)的主席团、部长团成员与理事长、常务理事必须为骨干成员。
    
    \item 满足下列任一条件则免除骨干成员:
    \begin{enumerate}
        \item 不再具有北京大学青年天文学会(学生社团)会员身份;
        \item 
    \end{enumerate}
\end{enumerate}

\section{高级会员}

\begin{enumerate}[resume]
    \item 考试每学期组织一次,可因特殊情况由主席团决定是否增加或减少考试次数。
    
    \item 考试由北大青天会会长组建“高级会员考试命题组”进行组织,若会长非高级会员,则由理事会推举一高级会员组建“高级会员考试命题组”。命题组全部组员必须是高级会员。
    
    \item 考试分为试题考试和观测考试,考试由命题组草拟题目、组织考试、阅卷评分、划定分数线。
    
    \item 考试内容为最基本的业余天文知识,适当参考历年天文奥赛题目、网络流行的天文题目以及样题,达到或超过分数线的同学方可通过。
    
    \item 所有普通会员均可参加考试,不限定次数。
\end{enumerate}

\section{会员的权利与义务}

\begin{enumerate}[resume]
    \item 所有会员可以收取青天会的活动通知短信和邮件。
    
    \item 所有会员有权利参加青天会组织的大型讲座、例行活动和外出观测。
    
    \item 所有会员有权利借用会内器材,出于对器材的保护原则,普通会员可以借用初级器材;高级会员可以借用初级器材和中级器材;高级器材需经主席团中一人同意后使用。原则上借用时间不超过两天。内联部设器材管理员,负责地下室器材的日常维护与分类。
    
    注:初级器材为双筒、寻星镜、目镜、巴洛镜、红光手电、电池等;中级器材为普通赤道仪和望远镜;高级器材为 heq3、heq5 两个赤道仪与小黑、80ed 两望远镜等常用设备;他人存放器材和古老的折反射望远镜不属于可借用器材;详细器材列表见《器材清单》。
    
    \item 会员有权利对学会的工作提出意见和建议。
\end{enumerate}

\section{会员证}

\begin{enumerate}[resume]
    \item 会员证是学会会员的身份凭证。
    
    \item 会员证分为两类:普通会员证和高级会员证,分别对应于普通会员和高级会员。
    
    \item 会员证在会员拥有北大学籍期间有效。
    
    \item 会员证的补办:会员证丢失可以向学会申请补办,补办需提供姓名、院系和学号。
\end{enumerate}

\section{附则}

\begin{enumerate}[resume]        
    \item 本条例未涉及部分,参照《北京大学学生社团管理办法》。???

    \item 本条例的解释权归北京大学青年天文学会(学生社团)所有,指导教师、挂靠单位和管理单位有权对本条例的解释提出建议。
\end{enumerate}
