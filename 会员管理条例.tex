\chapter{北京大学青年天文学会(学生社团)会员管理条例}

本条例于2019年x月xx日骨干成员会议讨论通过,自2019年x月xx日起正式施行,原《北京大学青年天文学会会员管理条例》自行废止。
为规范北京大学青年天文学会(学生社团)。。。。。。,根据《北京大学青年天文学会(学生社团)章程》制定本条例。

\section{总则}

\begin{enumerate}
    \item 北京大学青年天文学会为全体会员提供良好的业余天文交流平台,青年天文学会有义务为全体注册会员提供必要的支持和帮助。
    
    \item 北京大学青年天文学会例行会议有权对本条例进行修订和解释。
    
    \item 会员管理由学会内联部专门负责。
    
    \item 本条例未涉及部分,参照《北京大学社团管理章程》。
\end{enumerate}

\section{会员分类}

\begin{enumerate}[resume]
    \item 北京大学青年天文学会会员分为:普通会员、高级会员和骨干成员,其中骨干成员可以与其他两类会员有交集。
    
    \item 所有新加入的会员为普通会员,会员资格在其拥有北京大学学籍期间永久有效。会员若因毕业而失去北京大学在校学生身份,则将视为自动退出本学会。
    
    \item 新加入的会员需在招新时缴纳20元会费。
    
    \item 普通会员通过考核可以晋升为高级会员。
    
    \item 青年天文学会的理事会成员、会长、团支书、副会长和各部门部长、部员为学会骨干成员。
    
    \item 骨干成员身份由会员自愿报名、主席团批准后获得骨干身份。骨干成员有权随时提出放弃骨干身份。
    
    \item 骨干有参加例会和各种内部活动的权利,也有参加例会的义务。一学期三次以上无故不参加例会者自动丧失骨干权利。
\end{enumerate}

\section{高级会员考试}

\begin{enumerate}[resume]
    \item 考试每学期组织一次,可因特殊情况由主席团决定是否增加或减少考试次数。
    
    \item 考试由北大青天会会长组建“高级会员考试命题组” 进行组织,若会长非高级会员,则由理事会推举一高级会员组建“高级会员考试命题组”。命题组全部组员必须是高级会员。
    
    \item 考试分为试题考试和观测考试,考试由命题组草拟题目、组织考试、阅卷评分、划定分数线。
    
    \item 考试内容为最基本的业余天文知识,适当参考历年天文奥赛题目、网络流行的天文题目以及样题,达到或超过分数线的同学方可通过。
    
    \item 所有普通会员均可参加考试,不限定次数。
\end{enumerate}

\section{会员的权利}

\begin{enumerate}[resume]
    \item 所有会员可以收取青天会的活动通知短信和邮件。
    
    \item 所有会员有权利参加青天会组织的大型讲座、例行活动和外出观测。
    
    \item 所有会员有权利借用会内器材,出于对器材的保护原则,普通会员可以借用初级器材;高级会员可以借用初级器材和中级器材;高级器材需经主席团中一人同意后使用。原则上借用时间不超过两天。内联部设器材管理员,负责地下室器材的日常维护与分类。
    
    注:初级器材为双筒、寻星镜、目镜、巴洛镜、红光手电、电池等;中级器材为普通赤道仪和望远镜;高级器材为 heq3、heq5 两个赤道仪与小黑、80ed 两望远镜等常用设备;他人存放器材和古老的折反射望远镜不属于可借用器材;详细器材列表见《器材清单》。
    
    \item 会员有权利对学会的工作提出意见和建议。
\end{enumerate}

\section{会员证}

\begin{enumerate}[resume]
    \item 会员证是学会会员的身份凭证。
    
    \item 会员证分为两类:普通会员证和高级会员证,分别对应于普通会员和高级会员。
    
    \item 会员证在会员拥有北大学籍期间有效。
    
    \item 会员证的补办:会员证丢失可以向学会申请补办,补办需提供姓名、院系和学号。
\end{enumerate}
