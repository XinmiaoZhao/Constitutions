\chapter{北京大学青年天文学会(学生社团)骨干成员会议条例}

本条例于2019年x月xx日骨干成员会议讨论通过,自2019年x月xx日起正式施行,原《北京大学青年天文学会例会条例》自行废止。

为规范北京大学青年天文学会(学生社团)骨干成员会议流程,根据《北京大学青年天文学会(学生社团)章程》制定本条例。

\section{总则}

\begin{enumerate}
    \item 北京大学青年天文学会(学生社团)骨干成员会议是学会学会最高机构,日常代理行使会员权利。
    
    \item 原则上所有骨干成员均应参加骨干成员会议,如因故不能参加应事先请假。
\end{enumerate}

\section{会议时间与地点}

\begin{enumerate}[resume]
    \item 北京大学青年天文学会(学生社团)骨干成员会议在学期中原则上每周召开一次,时间一般为周日晚20:40,地点一般为理科二号楼地下一层2019W房间。
    
    \item 如遇放假等特殊情况,北京大学青年天文学会(学生社团)主席团有权更改骨干成员会议的召开频率、时间和地点。
    %补充更改时间地点的流程
\end{enumerate}

\section{会议通知与记录}

\begin{enumerate}[resume]
    \item 北京大学青年天文学会(学生社团)主席团成员应在骨干成员会议召开前通知所有骨干成员本次会议议题;若骨干成员会议的召开时间、地点出现变更,也应在原定的召开时间前通知。
    
    \item 北京大学青年天文学会(学生社团)骨干成员会议记录由学会组织部负责,主席团成员对会议记录有监督、审核的义务。
    
    \item 会议记录应包括下列内容:
    \begin{enumerate}
        \item 到场人员;
        \item 已完成的工作总结;
        \item 近期工作安排;
        \item 长期工作展望;
        \item 其他应当记录的内容。
    \end{enumerate}
    
    \item 会议记录应在骨干成员会议结束后以电子邮件形式向所有骨干成员以及指导教师、挂靠单位与管理单位负责人发送。
\end{enumerate}

\section{会议议程}

\begin{enumerate}[resume]
    \item 北京大学青年天文学会(学生社团)骨干成员会议的主持人一般为会长,若会长因故不能参加会议,则会长应提前确定主持人。
    
    \item !北京大学青年天文学会(学生社团)骨干成员会议按照下列流程进行:
    \begin{enumerate}
        \item 主持人宣布会议开始,组织部部长指定一人担任记录员。
        \item 主持人总结上一周的活动内容,针对其中的做得好的部分和做得不足的部分,骨干可以在此时发言讨论;
        \item 会长布置接下来一周的任务,安排相关人员的工作;
        \item 会长发言完毕,团支书发言,内容包括询问一些长期性工作的进展;
        \item 团支书发言完毕,副会长发言;
        \item 骨干就学会接下来的工作等展开讨论,最后会长做总结性发言;
        \item 主持人宣布会议结束,进入自由讨论时间。
    \end{enumerate}

    \item 如在骨干成员会议上进行表决,具有表决权的骨干成员按照《北京大学青年天文学会(学生社团)会员管理条例》第\ref{item:right_for_vote}条之规定确定。

    \item 如在骨干成员会议上召开选举大会,流程见《北京大学青年天文学会(学生社团)选举与换届办法》。
\end{enumerate}

\section{附则}

\begin{enumerate}[resume]
    \item 本条例的解释权归北京大学青年天文学会(学生社团)所有,指导教师、挂靠单位和管理单位有权对本条例的解释提出建议。
\end{enumerate}
