\chapter{北京大学青年天文学会职责分配}

\begin{enumerate}
    \item 会长:
    
    \begin{enumerate}
        \item 作为会内最高领导人,对外代表青年天文学会;
        \item 除观测、讲座外活动的主负责人;
        \item 制定学会工作计划,安排会内一个学期各项活动;
        \item 审批会内章程。
    \end{enumerate}
    
    \item 团支书:
    
    \begin{enumerate}
        \item 协助会长做好会内工作,会长不在时行使社团负责人职能;
        \item 会内工作的监督,活动部以及内联部活动监督。
    \end{enumerate}
    
    \item 副会长:
    
    \begin{enumerate}
        \item 协助会长及团支书完成工作,两者均不在时行使社团负责人职能;
        \item 作为机动力量处理会内各种突发事件;
        \item 管理学会财务,制定活动预算并记录学会各项开支;
        \item 监督学术部及宣传部工作。
    \end{enumerate}
    
    \item 各部门职责:
    
    \begin{enumerate}
        \item 学术部的职责是:
        
        \begin{enumerate}
            \item 组织会内培训;
            \item 进行青天指南的编写;
            \item 组织青天大观(科普推送)的编写;
            \item 组织高级会员考试;
        \end{enumerate}

        \item 活动部的职责是:
        
        \begin{enumerate}
            \item 组织各个活动小组的日常活动;
            \item 组织学术讲座;
            \item 组织例行观测;
        \end{enumerate}

        \item 宣传部的职责是:
        
        \begin{enumerate}
            \item 负责学会宣传海报的设计、制作、印刷、张贴和发放;
            \item 负责学会微信公众平台的运作,进行推送的制作与发布;
            \item 发布学术讲座信息和观测信息;
            \item 其他有关学会宣传方面的工作;
        \end{enumerate}

        \item 内联部的职责是:
        
        \begin{enumerate}
            \item 管理电子与纸质资料,器材与其它固定资产;
            \item 管理会内手册,包括会员名单,章程,工作经验;
            \item 组织会内骨干间的聚餐和聚会等活动;
            \item 例会时进行会议记录;
            \item 承担会长秘书职责;
        \end{enumerate}

        \item 外联部的职责是:
        
        \begin{enumerate}
            \item 负责与其他高校天文社团和天文组织的日常联络和感情维系;
            \item 组织与其他高校天文社团和天文组织的交流活动。
        \end{enumerate}

    \end{enumerate}
    
\end{enumerate}
