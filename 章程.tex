\chapter{北京大学青年天文学会(学生社团)章程}

本章程于2019年3月26日线上骨干会议讨论通过,经指导教师、挂靠单位与北京大学学生课外活动指导中心批准,自2019年3月27日起正式施行,原《北京大学青年天文学会章程》自行废止。

\section{总则}

第一条  北京大学青年天文学会(学生社团)是由热爱天文的北京大学学生自发组成的非营利性学术社团。

第二条  北京大学青年天文学会(学生社团)的中文简称是“北大青天会”,英文全称是“Peking University Youth Astronomy Society (Students Association)”,英文简称是“PKUYAS”。

第三条  北京大学青年天文学会(学生社团)的挂靠单位是共青团北京大学物理学院委员会,指导教师由北京大学物理学院天文学系教师担任。学会接受指导教师、挂靠单位和北京大学学生课外活动指导中心的管理和指导。

第四条  北京大学青年天文学会(学生社团)坚持以马克思列宁主义、毛泽东思想、邓小平理论、“三个代表”重要思想、科学发展观和习近平新时代中国特色社会主义思想为指导,坚持四项基本原则,严格遵守相关法律法规和学校规章制度及有关规定,关心国家的统一安全和民族的安定团结,维护学校正常的教学科研秩序与安定和谐的校园氛围。

第五条  北京大学青年天文学会(学生社团)的宗旨是:以会员为本,面向全校普及天文知识,为广大天文爱好者提供交流与学习的平台。

\section{学会活动}

第六条  北京大学青年天文学会(学生社团)组织进行下列活动:
(一)  校内例行观测;
(二)  外出观测;
(三)  会内培训;
(四)  学术讲座;
(五)  天文沙龙;
(六)  天文知识竞赛;
(七)  撰写科普文章;
(八)  参观北京天文馆以及北京地区附近的天文台;
(九)  组织和参与各地天文社团之间的交流;
(十)  其他各类促进天文学习和交流的活动。

\section{会员}

第七条  北京大学青年天文学会(学生社团)实行会员制。

第八条  北京大学青年天文学会(学生社团)会员必须是具有北京大学正式学籍的全日制在校学生,会员身份在会员身为具有北京大学正式学籍的全日制在校学生期间内有效。会员若因毕业、退学等原因失去北京大学全日制在校学生身份,则将被视为自动退出北京大学青年天文学会(学生社团)。

第九条  加入北京大学青年天文学会(学生社团)的条件:
(一)  自愿加入北京大学青年天文学会(学生社团);
(二)  遵守北京大学青年天文学会(学生社团)的规定;
(三)  身为具有北京大学正式学籍的全日制在校学生。

第十条  加入北京大学青年天文学会(学生社团)的程序:
(一)  每学期招新时现场申请加入或在学期中任何时间申请加入;
(二)  每位新入会会员缴纳20元会费,成为北京大学青年天文学会(学生社团)会员;
(三)  北京大学青年天文学会(学生社团)向新会员颁发会员证。

第十一条  北京大学青年天文学会(学生社团)由所有会员共有,所有会员平等地享有下列权利:
(一)  本学会的选举权、被选举权和表决权;
(二)  参加本学会活动的权利;
(三)  优先获得本学会服务的权利;
(四)  对本学会工作的批评建议权和监督权;
(五)  自由退出本学会的权利。

第十二条  北京大学青年天文学会(学生社团)会员应当履行下列义务:
(一)  积极维护本学会合法权益的义务;
(二)  按规定缴纳会费的义务;
(三)  积极参与学会活动的义务。

第十三条  北京大学青年天文学会(学生社团)会员分为:普通会员、高级会员和骨干成员。所有新加入北京大学青年天文学会(学生社团)的会员均为普通会员,普通会员可以通过考核(主要以高级会员考试的形式)晋升为高级会员。普通会员和高级会员都可申请成为骨干成员。高级会员与骨干成员的特殊权利与义务等规定见《北京大学青年天文学会(学生社团)会员管理条例》。

第十四条  北京大学青年天文学会(学生社团)会员如有严重违反本章程的行为,学会有开除其会员资格的权力。 

\section{组织机构和负责人}

第十五条  北京大学青年天文学会(学生社团)骨干成员会议是学会最高机构,日常代理行使会员权利。

第十六条  北京大学青年天文学会(学生社团)骨干成员会议行使下列职权:
(一)  解释和修订学会章程,监督学会章程的实施;
(二)  制定和修改学会各项办法条例;
(三)  选举学会主席团,根据主席团的提名,决定部长的人选;
(四)  审查和批准学会活动计划和计划执行情况的报告;
(五)  审查和批准学会的预算和预算执行情况的报告;
(六)  改变或者撤销主席团或各部门不适当的决定;
(七)  变更部门的建制;
(八)  决定与其他社团间关系的问题;
(九)  应当由学会最高机构行使的其他职权。

第十七条  北京大学青年天文学会(学生社团)骨干成员会议有权罢免下列人员:
(一)  学会会长、团支书、副会长;
(二)  学会理事长;
(三)  学会各部门部长、副部长。

第十八条  北京大学青年天文学会(学生社团)骨干成员会议在学期中原则上每周举办一次,具体组织形式见《北京大学青年天文学会(学生社团)骨干成员会议条例》。

第十九条  北京大学青年天文学会(学生社团)设置主席团,由会长、团支书、副会长组成。

第二十条  北京大学青年天文学会(学生社团)主席团的职责是协商决定学会各项事务的安排,对学会的发展共同负责。会长是北京大学青年天文学会(学生社团)的第一责任人,其职责是主持骨干成员会议,总体领导学会各项活动的组织和开展;团支书的职责是组织学会团支部的建设,协助会长组织学会的各项活动;副会长的职责是管理学会日常运营经费,协助会长和团支书组织学会的各项活动。

第二十一条  北京大学青年天文学会(学生社团)与指导教师、挂靠单位和北京大学学生课外活动指导中心的沟通交流主要通过主席团完成,重大活动、离京活动等的备案手续主要由会长负责完成。

第二十二条  北京大学青年天文学会(学生社团)会长、团支书、副会长的任期为一年,不得连任。每年春季学期末前由骨干成员会议选举产生下一届主席团,秋季学期前完成换届。具体的换届选举制度见《北京大学青年天文学会(学生社团)换届选举办法》。

第二十三条  北京大学青年天文学会(学生社团)设置理事会,成为骨干成员满两年或本科三年级以上且成为骨干成员满一年的离任骨干成员可以选择进入理事会。理事会设置理事长。

第二十四条  北京大学青年天文学会(学生社团)理事会是本学会的内部指导机构。理事会的职责是对学会的日常运行情况提供意见与建议,在学会出现突发重大事件时负责协商并维持本学会的基本运转。理事长的职责是组织理事会的建设,协调理事会成员间的意见。

第二十五条  北京大学青年天文学会(学生社团)理事长的任期为一年,不得连任。每年春季学期末前理事会讨论得出下一任理事长建议人选,经骨干成员会议表决通过后,秋季学期前完成换届。具体的换届制度见《北京大学青年天文学会(学生社团)换届选举办法》。

第二十六条  “突发重大事件”是指导致学会无法正常运行的需要立即处理的突发紧急事件,如主席团成员失联、重病、辞职、被公安机关逮捕等学会重大变故。具体处理方式见《北京大学青年天文学会(学生社团)突发重大事件处理办法》。

第二十七条  北京大学青年天文学会(学生社团)设置六个部门:观测部、宣传部、学术部、摄影部、外联部、组织部。各部门部长、副部长由主席团提名,经骨干成员会议通过后产生。各部门部长、副部长组成部长团。

第二十八条  北京大学青年天文学会(学生社团)的部门组织原则为主席团领导下的部长负责制。

第二十九条  观测部的职责是:
(一)  组织例行观测;
(二)  承担外出观测活动中的天文观测部分;
(三)  负责学会望远镜等天文设备器材的购置、管理与维修;
(四)  其他与学会观测活动有关的工作。

第三十条  宣传部的职责是:
(一)  维护学会的官方微信公众号,进行信息的发布与升级;
(二)  负责学会宣传海报的设计,印刷;
(三)  发布学术讲座信息和观测信息;
(四)  其他与学会宣传活动有关的工作。

第三十一条  学术部的职责是:
(一)  组织会内培训;
(二)  组织学术讲座;
(三)  编写和修订《青天指南》;
(四)  撰写学会官方微信公众号的科普推送;
(五)  其他与学会学术活动有关的工作。

第三十二条  摄影部的职责是:
(一)  组织会内摄影培训;
(二)  为学会官方微信公众号提供天文摄影图片;
(三)  其他与学会天文摄影活动有关的工作。

第三十三条  外联部的职责是:
(一)  负责学会与北京大学各个学生社团的交流与合作;
(二)  负责学会与各高校天文社团、中小学校天文社团、天文爱好者社会团体的交流与合作;
(三)  学会的其他对外联络工作。

第三十四条  组织部的职责是:
(一)  管理电子与纸质资料,包括会员名单,学会章程与管理办法,工作备忘录等;
(二)  负责学会除望远镜等天文设备器材以外的设备资产的购置、管理与维修;
(三)  协助其他部门组织学会的各项活动;
(四)  组织会内骨干间的聚餐和聚会等活动;
(五)  组织骨干成员会议的会议记录、审核、发送与上报;
(六)  学会的其他组织后勤工作。

\section{资产的管理与使用}

第三十五条  北京大学青年天文学会(学生社团)的资产包括学会日常运营经费和设备资产,设备资产指的是望远镜等天文设备器材与其他设备器材。

第三十六条  北京大学青年天文学会(学生社团)的资产不可侵犯,禁止任何组织或者个人采用任何手段侵占或破坏学会的资产。

第三十七条  北京大学青年天文学会(学生社团)的日常运营经费包括下列来源:
(一)  会员缴纳的会费;
(二)  社团发展基金;
(三)  院系团委资助;
(四)  活动经费盈余;
(五)  捐赠与赞助。

第三十八条  北京大学青年天文学会(学生社团)的日常运营经费应用于学会日常开销和活动的筹备与举办,禁止在骨干成员或会员中直接分配。

第三十九条  北京大学青年天文学会(学生社团)日常运营经费的管理由副会长负责,副会长应建立完善的财务登记薄,保证会计资料合法、真实、准确、完整。

第四十条  北京大学青年天文学会(学生社团)的望远镜等天文设备器材必须在经过主席团考核的骨干成员的看护下使用,看护人员必须向使用者叙述操作规范,尽量避免人为损坏。如有损坏,本学会保有向损坏者收取适当修理费用的权利。

\section{会徽、会旗、会歌}

第四十一条  北京大学青年天文学会(学生社团)的会徽为:上部三角形内,上方是金色四角星,下方是“YAS”方形字样,“YAS”颜色依次是绿色、蓝色、红色;三角形下三行黑色字分别是“PKU”、 “Youth Astronomy Society”、“Since 1990”,其中“Y”、“A”、“S”三个字母颜色依次是绿色、蓝色、红色。

第四十二条  北京大学青年天文学会(学生社团)的会旗为:左上是北京大学青年天文学会(学生社团)会徽,右上是“北京大学”标准红色字样,下部是“青年天文学会”蓝色华文新魏字样。

第四十三条  北京大学青年天文学会(学生社团)的会歌为《跟我一起学天文》。

\section{附则}

第四十四条  本章程以及学会各项办法条例的修订,需由主席团或三位以上骨干成员发起,由骨干成员会议以全体骨干成员的三分之二以上的多数通过,并经指导教师、挂靠单位与北京大学学生课外活动指导中心批准后方可生效。

第四十五条  本章程的解释权归北京大学青年天文学会(学生社团)所有,指导教师、挂靠单位和北京大学学生课外活动指导中心有权对本章程的解释提出建议。
