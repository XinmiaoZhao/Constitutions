\chapter{北京大学青年天文学会(学生社团)章程}

本章程于2019年x月xx日骨干成员会议讨论通过,经指导教师、挂靠单位与管理单位批准,自2019年x月xx日起正式施行,原《北京大学青年天文学会(学生社团)章程》自行废止。

\section{总则}

\begin{enumerate}
    \item 北京大学青年天文学会(学生社团)是由热爱天文的北京大学学生自发组成的非营利性学术社团。
    
    \item 北京大学青年天文学会(学生社团)的中文简称是“北大青天会”,英文全称是“Peking University Youth Astronomy Society (Students Association)”,英文简称是“PKUYAS”。
    
    \item 北京大学青年天文学会(学生社团)的管理单位是共青团北京大学委员会,挂靠单位是共青团北京大学物理学院委员会,指导教师由北京大学物理学院天文学系教师担任。学会接受指导教师、挂靠单位和管理单位的管理和指导。
    
    \item 北京大学青年天文学会(学生社团)坚持以马克思列宁主义、毛泽东思想、邓小平理论、“三个代表”重要思想、科学发展观和习近平新时代中国特色社会主义思想为指导,坚持四项基本原则,严格遵守相关法律法规和学校规章制度及有关规定,关心国家的统一安全和民族的安定团结,维护学校正常的教学科研秩序与安定和谐的校园氛围。
    
    \item 北京大学青年天文学会(学生社团)的宗旨是:以会员为本,面向全校普及天文知识,为广大天文爱好者提供交流与学习的平台。
\end{enumerate}

\section{学会活动}

\begin{enumerate}[resume]
    \item 北京大学青年天文学会(学生社团)组织进行下列活动:
    
    \begin{enumerate}
        \item 校内例行观测;
        \item 外出观测;
        \item 会内培训;
        \item 学术讲座;
        \item 天文沙龙;
        \item 天文知识竞赛;
        \item 撰写科普文章;
        \item 参观北京天文馆以及北京地区附近的天文台;
        \item 组织和参与各地天文社团之间的交流;
        \item 其他各类促进天文学习和交流的活动。
    \end{enumerate}
    
\end{enumerate}

\section{会员}

\begin{enumerate}[resume]
    \item 北京大学青年天文学会(学生社团)实行会员制,加入学会的学生称为北京大学青年天文学会(学生社团)会员,具有北京大学青年天文学会(学生社团)会籍。
    
    \item 北京大学青年天文学会(学生社团)会员必须是具有北京大学正式学籍的在校学生,会员身份在会员身为具有北京大学正式学籍的在校学生期间内有效。会员若因毕业、退学等原因失去北京大学全日制在校学生身份,则将被视为自动退出北京大学青年天文学会(学生社团)。
    
    \item 北京大学青年天文学会(学生社团)会员分为:普通会员、高级会员和骨干成员。所有新加入北京大学青年天文学会(学生社团)的会员均为普通会员,普通会员可以通过考核(主要以高级会员考试的形式)晋升为高级会员。普通会员和高级会员都可申请成为骨干成员。
    
    \item 满足下列条件的具有北京大学正式学籍的在校学生可以自愿申请加入北京大学青年天文学会(学生社团):
    
    \begin{enumerate}
        \item 认同北京大学青年天文学会(学生社团)的宗旨;
        \item 愿意参加北京大学青年天文学会(学生社团)组织的活动;
        \item 遵守学会章程以及各项办法条例。
    \end{enumerate}
    
    \item 北京大学青年天文学会(学生社团)所有会员平等地享有下列权利:
    
    \begin{enumerate}
        \item 获得学会活动通知的权利;
        \item 参加学会活动的权利;
        \item ?优先获得学会服务的权利;
        \item 对学会工作提出批评、意见或建议,进行监督的权利;
        \item 自由退出学会的权利。
    \end{enumerate}
    
    \item 北京大学青年天文学会(学生社团)会员应当履行下列义务:
    
    \begin{enumerate}
        \item 积极维护学会合法权益的义务;
        \item 按规定缴纳会费的义务;
        \item 积极参加学会活动的义务。
    \end{enumerate}

    \item 有关会员管理的具体规定见《北京大学青年天文学会(学生社团)会员管理条例》。
    
\end{enumerate}

\section{组织机构和负责人}

\begin{enumerate}[resume]
    \item 北京大学青年天文学会(学生社团)骨干成员会议是学会最高机构,?日常代理行使会员权利。
    
    \item 北京大学青年天文学会(学生社团)骨干成员会议行使下列职权:
    
    \begin{enumerate}
        \item 解释和修订学会章程,监督学会章程的实施;
        \item 制定和修改学会各项办法条例;
        \item 选举学会主席团,根据主席团的提名,决定部长团的人选;
        \item 审查和批准学会活动计划和计划执行情况的报告;
        \item 审查和批准学会的预算和预算执行情况的报告;
        \item 改变或者撤销主席团或各部门不适当的决定;
        \item 变更部门的建制;
        \item 决定与其他社团间关系的问题;
        \item 应当由学会最高机构行使的其他职权。
    \end{enumerate}
    
    \item 北京大学青年天文学会(学生社团)骨干成员会议有权罢免下列人员:
    
    \begin{enumerate}
        \item 学会会长、团支书、秘书长;
        \item 学会副会长;
        \item 学会各部门部长、副部长;
        \item 学会理事长。
    \end{enumerate}
    
    \item 北京大学青年天文学会(学生社团)骨干成员会议在学期中原则上每周举办一次,具体组织形式见《北京大学青年天文学会(学生社团)骨干成员会议条例》。
    
    \item 北京大学青年天文学会(学生社团)设置主席团,由会长、团支书、秘书长组成。主席团成员为社团负责人。
    
    \item 北京大学青年天文学会(学生社团)主席团的职责是协商决定学会各项事务的安排,对学会的发展共同负责。会长是北京大学青年天文学会(学生社团)的第一责任人,其职责是主持骨干成员会议,总体领导学会各项活动的组织和开展;团支书的职责是组织学会团支部的建设,协助会长组织学会的各项活动;秘书长的职责是协助会长和团支书组织学会的各项活动。
    
    \item 北京大学青年天文学会(学生社团)与指导教师、挂靠单位和管理单位的沟通交流主要通过主席团完成,重大活动、离京活动等的备案手续主要由会长负责完成。
    
    \item 北京大学青年天文学会(学生社团)会长、团支书、秘书长的任期为一年,不得连任。每年春季学期末前由骨干成员会议选举产生下一届主席团,秋季学期前完成换届。具体的选举与换届办法见《北京大学青年天文学会(学生社团)选举与换届办法》。
    
    \item !北京大学青年天文学会(学生社团)设置一至三名副会长,由主席团提名,经骨干成员会议通过后产生。副会长的职责是协助主席团组织学会的各项活动。主席团指定一名副会长为财务负责人,负责管理学会日常运营经费。
    
    \item 北京大学青年天文学会(学生社团)设置六个部门,分别为观测部、宣传部、学术部、摄影部、外联部、组织部。各部门部长、副部长由主席团提名,经骨干成员会议通过后产生。副会长、各部门部长、副部长组成部长团。
    
    \item 北京大学青年天文学会(学生社团)的部门组织原则为主席团领导下的部长负责制。各部门职责见《北京大学青年天文学会(学生社团)部门管理办法》。
    
    \item 北京大学青年天文学会(学生社团)设置理事会,离任主席团成员、部长团成员以及成为骨干成员满两年或本科三年级以上且成为骨干成员满一年的离任骨干成员可以选择进入理事会。理事会设置理事长。理事会的管理办法见《北京大学青年天文学会(学生社团)理事会管理办法》。
    
    \item 北京大学青年天文学会(学生社团)理事会是学会的内部指导机构。理事会的职责是对学会的日常运行情况提供意见与建议,在学会出现突发重大事件时负责协商并维持学会的基本运转。理事长的职责是组织理事会的建设,协调理事会成员间的意见。
    
    \item 北京大学青年天文学会(学生社团)理事长的任期为一年,不得连任。每年春季学期末前理事会讨论得出下一任理事长建议人选,经骨干成员会议表决通过后完成换届。具体的换届制度见《北京大学青年天文学会(学生社团)选举与换届办法》。
    
    \item “突发重大事件”是指导致学会无法正常运行的需要立即处理的突发紧急事件,如主席团成员失联、重病、辞职、被公安机关逮捕等学会重大变故。具体处理方式见《北京大学青年天文学会(学生社团)突发重大事件处理办法》。
    
\end{enumerate}

\section{资产的管理与使用}

\begin{enumerate}[resume]
    \item 北京大学青年天文学会(学生社团)的资产包括学会日常运营经费和设备资产,设备资产指的是望远镜等天文设备器材与其他设备器材。
    
    \item 北京大学青年天文学会(学生社团)的资产不可侵犯,禁止任何组织或者个人采用任何手段侵占或破坏学会的资产。
    
    \item 北京大学青年天文学会(学生社团)的日常运营经费包括下列来源:
    
    \begin{enumerate}
        \item 会员缴纳的会费;
        \item 社团发展基金;
        \item 院系团委资助;
        \item 活动经费盈余;
        \item 捐赠与赞助。
    \end{enumerate}
    
    \item 北京大学青年天文学会(学生社团)的日常运营经费应用于学会日常开销和活动的筹备与举办,禁止在骨干成员或会员中直接分配。
    
    \item 北京大学青年天文学会(学生社团)日常运营经费的管理由财务负责人负责,财务负责人应建立完善的财务登记薄,保证会计资料合法、真实、准确、完整。
    
    \item 北京大学青年天文学会(学生社团)的望远镜等天文设备器材必须在经过主席团考核的骨干成员的看护下使用,看护人员必须向使用者叙述操作规范,尽量避免人为损坏。如有损坏,学会保有向损坏者收取适当修理费用的权利。
\end{enumerate}

\section{会徽、会旗、会歌}

\begin{enumerate}[resume]
    \item 北京大学青年天文学会(学生社团)的会徽为:上部三角形内,上方是金色四角星,下方是“YAS”方形字样,“YAS”颜色依次是绿色、蓝色、红色;三角形下三行黑色字分别是“PKU”、 “Youth Astronomy Society”、“Since 1990”,其中“Y”、“A”、“S”三个字母颜色依次是绿色、蓝色、红色。
    
    \item 北京大学青年天文学会(学生社团)的会旗为:左上是北京大学青年天文学会(学生社团)会徽,右上是“北京大学”标准红色字样,下部是“青年天文学会”蓝色华文新魏字样。
    
    \item 北京大学青年天文学会(学生社团)的会歌为《跟我一起学天文》。
\end{enumerate}

\section{附则}

\begin{enumerate}[resume]
    \item 本章程以及学会各项办法条例的修订,需由主席团或三位以上骨干成员发起,由骨干成员会议以全体具有表决权的骨干成员的三分之二以上的多数通过,并经指导教师、挂靠单位与管理单位批准后方可生效。
    
    \item 本章程的解释权归北京大学青年天文学会(学生社团)所有,指导教师、挂靠单位和管理单位有权对本章程的解释提出建议。
\end{enumerate}